\documentclass[a4paper,11pt]{article}
\usepackage[english]{babel}
% Para Windows
% \usepackage[utf8]{inputenc} %substituir 'latin1' por 'utf8' para editores ShareLaTex ou Overleaf
% \usepackage[T1]{fontenc}
% Para Linux
\usepackage[utf8]{inputenc}
% Para Mac
% \usepackage[applemac]{inputenc}
% ---]
% Packages [---
\usepackage{fancyhdr} %para cope�alho e rodap�
\usepackage{fancybox} %para cope�alho e rodap�
\pagestyle{fancy} %para cope�alho e rodap�
\usepackage{easylist} %para aninhar enumerate
\usepackage{paralist} %para compactenum
\usepackage{pbox} %para parbox
\usepackage[hyphens]{url}  %para url com line break
\usepackage{multirow} %para multirow e multicolumn
\usepackage{array, hhline} %mais opcoes para o ambiente tabular
\usepackage{fourier} %symbols
\usepackage{ifsym} %symbols
\usepackage{graphicx} %para inserir jpg e png e pdf
\usepackage{float}
\usepackage{longtable}
% ---]

% Cria hyperlinks e estrutura de t�picos para navegar em um pdf reader [---
% N�o alterar
\usepackage[pdftex]{hyperref}
\usepackage{color}
\definecolor{corlinkprint}{rgb}{0, 0, 0}
\hypersetup{
  colorlinks,
  citecolor   = corlinkprint,
  filecolor   = corlinkprint,
  linkcolor   = corlinkprint,
  urlcolor    = corlinkprint}
% ---]

% Cabe�alho e rodap� a partir da p�gina 2 [---
% Substituir 'Modelo' pelo nome do projeto (nome de uma palavra usado na disciplina)
\fancyhead[L]{\small{Technical Report - SPARtS}}
\fancyhead[R]{\small\thepage} 
\fancyfoot{}
\renewcommand\headrulewidth{0.4pt}
% ---]

% T�tulo [---
% Substituir 'Modelo...' pelo t�tulo do relat�rio
\title{
Technical Report\\
\textbf{SPARtS - Small Parts Automated Retrieval and Storage}\\}
% ---]

% Autores [---
% Substituir pelos nomes e emails dos integrantes
\author{
Gustavo F. Armênio 1 \footnotesize{-- emailpessoal1@qqrlugar.com}\\
Ian M. S. Ishikawa 2 \footnotesize{-- ianishikawa@alunos.utfpr.edu.br}\\
Lucas W. Nascimento 3 \footnotesize{-- lucaswalger@alunos.utfpr.edu.br}\\
Otavio L. S. Pepe 4 \footnotesize{-- emailpessoaln@qqrlugar.com}\\
}
% ---]

% Substituir pelo mes e ano da defesa [---
\date{November 2025}
% ---]

\begin{document}

% Cabe�alho da p�gina 1 [---
% N�o alterar
\thisfancyput(35mm,-10mm){
\begin{tabular}{c}
Federal University of Technology - Paraná - Brazil -- \small{UTFPR}\\
Academic Department of Electronics -- \small{DAELN}\\
Academic Department of Informatics -- \small{DAINF}\\
Computer Engineering\\
Integration Workshop 3 (ELEX22) -- S71 -- 2025/2 \\
\\
\hline
\end{tabular}}
% ---]

% Insere t�tulo, autores e data [---
\maketitle
% ---]

% Resumo [---
% Sustituir pelo seu resumo
\begin{abstract}
\noindent This report presents SPARtS, an automated storage and retrieval system designed to organize small parts in workshops and laboratories. The project addresses common challenges such as the time-consuming process of locating components, the high number of similar items, and the possibility of human error when storing parts. SPARtS integrates mechanical, hardware, and software components to automatically store and retrieve items through menu browsing, text search, or visual identification, and keep real-time stock estimates. The mechanical structure includes an XYZ movement system, a conveyor belt, a retractable platform, and 24-slot storage built with MDF, aluminum extrusions, and 3D-printed parts. The hardware architecture employs an ESP32 microcontroller, ESP32-CAM, load cell with HX711, RFID sensors, stepper motors, and custom-made circuitry. The software system combines a Vue.js frontend, a Python backend, and a YOLOv11-based computer vision model capable of identifying multiple item classes with high reliability. The system meets all functional requirements, demonstrating accurate positioning, reliable stock estimation, intuitive user interaction, and effective detection of item classes. With optimized workflow and precise automation, SPARtS proves to be a successful and efficient solution for improving organization and operational speed in workshop environments.
\end{abstract}
% ---]

%===============================================================
% Corpo do texto [---
% Substituir pelo seu relat�rio

\section{Introduction}
\label{sec:intro}

\noindent
Workshops and laboratories frequently rely on a large variety of small parts for daily operations. However, storing, locating, and managing these components can become a time-consuming and error-prone task, especially when many similar items are involved. Human mistakes in placing items in incorrect locations, combined with the difficulty of maintaining an accurate inventory, highlight the need for a more efficient organizational solution.

To address these challenges, the SPARtS project proposes an automated system capable of storing, retrieving, organizing, and tracking small parts with minimal human intervention. The system integrates mechanical automation, embedded hardware, computer vision, and a user-friendly software interface to enable fast and reliable handling items. By combining an XYZ movement structure, a conveyor-based insertion mechanism, a load-cell-based stock estimation system, and YOLOv11-powered visual identification, SPARtS significantly improves operational efficiency in workshop environments.

This work presents the design, implementation, and evaluation of SPARtS, demonstrating how automation and intelligent processing can streamline inventory management and support more organized, accurate, and productive workflows.

\section{Project Specification}
\label{specification}

\subsection{Requirements}
\label{requirements}

This section presents the main requirements established by the SPARtS project team during the planning phase, taking into account the available resources and the intended system functionalities. The requirements were divided into functional and non-functional categories and organized according to the system’s subsystems: software, mechanical, and hardware.

\subsection{Mechanical Requirements}
\label{mech_requirements}

\begin{table}[h!]
\centering
\begin{tabular}{|l|l|p{8cm}|}
\hline
Category & Requirement ID & Description \\ \hline

Mechanical & FR01  &  The system must store up to 24 storage bins each with a maximum of 100 grams of parts, for parts with minimum individual weight of 2 grams and maximum individual dimension in any direction of 30 mm \\ \hline

Mechanical & FR02  & The system must deliver a retrieved storage bin in a output compartment \\ \hline

Mechanical & FR03  &  The system must fetch and move bins across all storage areas and the user output area\\ \hline

Mechanical & FR04  & The system must have a mechanism for automated storage of parts separated in buffer bins\\ \hline

Mechanical & FR05  & The system must allow user manual access to the bins \\ \hline

\end{tabular}
\caption{Mechanical Requirements}
\end{table}

\subsection{Hardware Requirements}
\label{hw_requirements}

\begin{table}[h!]
\centering
\begin{tabular}{|l|l|p{8cm}|}
\hline
Category & Requirement ID & Description \\ \hline

Hardware & FR06  &  The system must be able to verify the ID of a bin with an individual ID tag \\ \hline

Hardware & FR07  & The system must capture images of items in a buffer bin \\ \hline

Hardware & FR08  &  The system must estimate stock by measuring the weight of a storage bin in the output compartment with a margin of error of +-3 grams\\ \hline

Hardware & FR09  & The system must have a main On/Off switch\\ \hline

\end{tabular}
\caption{Hardware Requirements}
\end{table}

\subsection{Software Requirements}
\label{sw_requirements}


\begin{longtable}{|l|l|p{8cm}|}

\hline
Category & Requirement ID & Description \\ \hline

Software & FR010  &  The embedded software must retrieve an item’s bin from the storage in at most 45 seconds if required by the application and the item bin is in the expected placement \\ \hline

Software & FR011 & The embedded software must return an item’s bin to the storage in at most 30 seconds if required by the application and there is no unexpected bin in its place \\ \hline

Software & FR012 &  The system must have an application for user interface\\ \hline

Software & FR013 & The application must allow the user to select an item for retrieval\\ \hline

Software & FR014 & The application must have an option of automatic storage\\ \hline

Software & FR015 & The system must identify wrong storage bin placements\\ \hline

Software & FR016 & The system must be able to reorganize storage bin placements\\ \hline

Software & FR017 & The embedded software must check the ID of a storage bin if required by the application\\ \hline

Software & FR018 & The system must update the stock estimation of a bin whenever it weights that storage bin\\ \hline

Software & FR019 & The system should be able to visually identify 12 selected parts including screws, nuts, and corner supports with a minimum of 50\%\ mAP\\ \hline

Software & FR020 & The stock estimation must have a precision of +/-4 items\\ \hline

\caption{Software Requirements}\\
\end{longtable}


\section{Project Development}
\label{development}

\subsection{Mechanical Development}
\label{mech_development}

\subsection{Hardware Development}
\label{hw_development}

\subsection{Software Development}
\label{sw_development}

\section{Results}
\label{results}

\subsection{Budget}
\label{budget}

\subsection{Fucntional requirements completion}
\label{func_requirements_comp}

\section{Conclusion}
\label{conclusion}



%Insere referencias [---
\bibliographystyle{unsrt}
\bibliography{IF66J_v1} % Substituir pelo seu arquivo bib
% ---]

% ---]

\end{document}

