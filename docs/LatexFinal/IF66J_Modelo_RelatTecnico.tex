\documentclass[a4paper,11pt]{article}
\usepackage[english]{babel}
% Para Windows
% \usepackage[utf8]{inputenc} %substituir 'latin1' por 'utf8' para editores ShareLaTex ou Overleaf
% \usepackage[T1]{fontenc}
% Para Linux
\usepackage[utf8]{inputenc}
% Para Mac
% \usepackage[applemac]{inputenc}
% ---]
% Packages [---
\usepackage{fancyhdr} %para cope�alho e rodap�
\usepackage{fancybox} %para cope�alho e rodap�
\pagestyle{fancy} %para cope�alho e rodap�
\usepackage{easylist} %para aninhar enumerate
\usepackage{paralist} %para compactenum
\usepackage{pbox} %para parbox
\usepackage[hyphens]{url}  %para url com line break
\usepackage{multirow} %para multirow e multicolumn
\usepackage{array, hhline} %mais opcoes para o ambiente tabular
\usepackage{fourier} %symbols
\usepackage{ifsym} %symbols
\usepackage{graphicx} %para inserir jpg e png e pdf
% ---]

% Cria hyperlinks e estrutura de t�picos para navegar em um pdf reader [---
% N�o alterar
\usepackage[pdftex]{hyperref}
\usepackage{color}
\definecolor{corlinkprint}{rgb}{0, 0, 0}
\hypersetup{
  colorlinks,
  citecolor   = corlinkprint,
  filecolor   = corlinkprint,
  linkcolor   = corlinkprint,
  urlcolor    = corlinkprint}
% ---]

% Cabe�alho e rodap� a partir da p�gina 2 [---
% Substituir 'Modelo' pelo nome do projeto (nome de uma palavra usado na disciplina)
\fancyhead[L]{\small{Technical Report - SPARtS}}
\fancyhead[R]{\small\thepage} 
\fancyfoot{}
\renewcommand\headrulewidth{0.4pt}
% ---]

% T�tulo [---
% Substituir 'Modelo...' pelo t�tulo do relat�rio
\title{
Technical Report\\
\textbf{SPARtS - Small Parts Automated Retrieval System}\\}
% ---]

% Autores [---
% Substituir pelos nomes e emails dos integrantes
\author{
Gustavo F. Armênio 1 \footnotesize{-- emailpessoal1@qqrlugar.com}\\
Ian M. S. Ishikawa 2 \footnotesize{-- ianishikawa@alunos.utfpr.edu.br}\\
Lucas W. Nascimento 3 \footnotesize{-- lucaswalger@alunos.utfpr.edu.br}\\
Otavio L. S. Pepe 4 \footnotesize{-- emailpessoaln@qqrlugar.com}\\
}
% ---]

% Substituir pelo mes e ano da defesa [---
\date{November 2025}
% ---]

\begin{document}

% Cabe�alho da p�gina 1 [---
% N�o alterar
\thisfancyput(35mm,-10mm){
\begin{tabular}{c}
Federal University of Technology - Paraná - Brazil -- \small{UTFPR}\\
Academic Department of Electronics -- \small{DAELN}\\
Academic Department of Informatics -- \small{DAINF}\\
Computer Engineering\\
Integration Workshop 3 (ELEX22) -- S71 -- 2025/2 \\
\\
\hline
\end{tabular}}
% ---]

% Insere t�tulo, autores e data [---
\maketitle
% ---]

% Resumo [---
% Sustituir pelo seu resumo
\begin{abstract}
\noindent This report presents SPARtS, an automated storage and retrieval system designed to organize small parts in workshops and laboratories. The project addresses common challenges such as the time-consuming process of locating components, the high number of similar items, and the possibility of human error when storing parts. SPARtS integrates mechanical, hardware, and software components to automatically store and retrieve items through menu browsing, text search, or visual identification, and keep real-time stock estimates. The mechanical structure includes an XYZ movement system, a conveyor belt, a retractable platform, and 24-slot storage built with MDF, aluminum extrusions, and 3D-printed parts. The hardware architecture employs an ESP32 microcontroller, ESP32-CAM, load cell with HX711, RFID sensors, stepper motors, and custom-made circuitry. The software system combines a Vue.js frontend, a Python backend, and a YOLOv11-based computer vision model capable of identifying multiple item classes with high reliability. The system meets all functional requirements, demonstrating accurate positioning, reliable stock estimation, intuitive user interaction, and effective detection of item classes. With optimized workflow and precise automation, SPARtS proves to be a successful and efficient solution for improving organization and operational speed in workshop environments.
\end{abstract}
% ---]

%===============================================================
% Corpo do texto [---
% Substituir pelo seu relat�rio

\section{Introduction}
\label{sec:intro}

\noindent
Workshops and laboratories frequently rely on a large variety of small parts for daily operations. However, storing, locating, and managing these components can become a time-consuming and error-prone task, especially when many similar items are involved. Human mistakes in placing items in incorrect locations, combined with the difficulty of maintaining an accurate inventory, highlight the need for a more efficient organizational solution.

To address these challenges, the SPARtS project proposes an automated system capable of storing, retrieving, organizing, and tracking small parts with minimal human intervention. The system integrates mechanical automation, embedded hardware, computer vision, and a user-friendly software interface to enable fast and reliable handling items. By combining an XYZ movement structure, a conveyor-based insertion mechanism, a load-cell-based stock estimation system, and YOLOv11-powered visual identification, SPARtS significantly improves operational efficiency in workshop environments.

This work presents the design, implementation, and evaluation of SPARtS, demonstrating how automation and intelligent processing can streamline inventory management and support more organized, accurate, and productive workflows.

\subsection{Proposed Solution}
\label{sec:solution}

\noindent
In order to solve the problems presented in section \ref{sec:intro}, SPARtS follow the following sequence. 

\section{Project Specification}
\label{specification}

\subsection{Requirements}
\label{requirements}

\subsection{Mechanical Requirements}
\label{mech_requirements}

\subsection{Hardware Requirements}
\label{hw_requirements}

\subsection{Software Requirements}
\label{sw_requirements}

\section{Project Development}
\label{development}

\subsection{Mechanical Development}
\label{mech_development}

\subsection{Hardware Development}
\label{hw_development}

\subsection{Software Development}
\label{sw_development}

\section{Results}
\label{results}

\subsection{Budget}
\label{budget}

\subsection{Fucntional requirements completion}
\label{func_requirements_comp}

\section{Conclusion}
\label{conclusion}



%Insere referencias [---
\bibliographystyle{unsrt}
\bibliography{IF66J_v1} % Substituir pelo seu arquivo bib
% ---]

% ---]

\end{document}

